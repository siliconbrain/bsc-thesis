%----------------------------------------------------------------------------
\chapter{Összefoglalás}
\label{chap:summary}
%----------------------------------------------------------------------------

A szakdolgozat elkészítése során megismerkedtem a modell-vezérelt szoftverfejlesztéssel, és az azt támogató Eclipse Modeling Framework-kel és Eclipse integrált fejlesztői környezettel.
Megismertem továbbá az EMF-IncQuery, gráfminta alapú modell-lekérdezések definiálását és végrehajtását lehetővé tevő keretrendszerrel.
Elemeztem az EMF-IncQuery lekérdezőnyelvében statikus analízis segítségével megoldható problémák témakörét, és kiválasztottam ezek közül kettőt: a változók használatával kapcsolatos problémákat (elgépelés, egyszeri hivatkozás, \ldots), illetve a származtatott tulajdonságok inkonzisztens jelleg-megadásának problémáját.
A problémák okainak és körülményeinek mélyebb megismerése után megterveztem és megvalósítottam olyan statikus analízis alapú ellenőrzéseket, melyek fejlesztési időben hívják fel a fejlesztők figyelmét ezekre a problémákra.
Végül az eredetileg az ellenőrzésekhez elkészített, funkcionális függőségeket elemző algoritmusokat felhasználtam a lekérdezések futási teljesítményének javításához.

A dolgozat elkészítése folyamán lehetőségem adódott hozzájárulni egy hivatalos, nyílt forrású Eclipse projekt -- az EMF-IncQuery -- fejlődéséhez.
Az EMF-IncQuery fejlesztői és felhasználói végig érdeklődést mutattak a készülő ellenőrzések iránt, és használatba is vették azokat, mihelyst elkészültek.

Röviden összefoglalva, az alábbi célokat értem el:
\begin{itemize}
    \item megismertem és a dolgozatban bemutattam modell-vezérelt szoftverfejlesztés és a modell-lekérdezések fogalmát az EMF-IncQuery keretrendszeren keresztül 
    \item megterveztem és megvalósítottam kétféle, statikus analízisen alapuló ellenőrzést az EMF-IncQuery gráfminták leírására szolgáló nyelvéhez
    \item az egyik ilyen ellenőrzéshez készült, funkcionális függőségek elemzését végző kódot az EMF-IncQuery fejlesztőivel közösen felhasználtuk a lekérdezések kiértékelési teljesítményének növeléséhez, melyet egy rövid példán be is mutattam
\end{itemize}

Az általam készített ellenőrzések forrásai megtalálhatóak az alábbi online tárhelyeken:
\begin{itemize}
\item \textit{változók használatának vizsgálata:} \url{https://github.com/siliconbrain/EMF-IncQuery/tree/unused-variable-check}
\item \textit{funkcionális függőség alapú vizsgálatok:} \url{https://github.com/siliconbrain/org.eclipse.incquery/tree/func-dep}
\end{itemize}

\section{Kitekintés}

Az elkészült ellenőrzések ugyan jelen állapotukban is sok segítséget nyújtanak, ám adódik még lehetőség további fejlesztésükre.

A változók használatának vizsgálata nem tér ki a Java kódot tartalmazó kényszerekre, ám ez a hiányossága például egy önálló labormunka keretében pótolható.

A funkcionális függőség alapú vizsgálatok több ponton is kiegészíthetőek.
Az egyik ilyen lehetőség a függőségi kapcsolatok minőségének bevezetésével a sok törzsű minták esetének kezelése.
Egy másik lehetőség a kényszerekből függőségeiket következtető funkcionalitás lecserélése az EMF-IncQuery PSystem-jében már meglévő algoritmusra, ezzel csökkentve a karbantartandó kód mennyiségét.

Végül, az általam készítetteken kívül még rengeteg más statikus ellenőrzés is elképzelhető az EMF-IncQuery lekérdezőnyelvéhez, vagy akár a keretrendszer többi részéhez is, mind emberi mulasztásból fakadó hibák megelőzésére, mind a végrehajtási teljesítmény javítására. 
