%----------------------------------------------------------------------------
\chapter{Összefoglalás}
\label{chap:summary}
%----------------------------------------------------------------------------

A modell-vezérelt szoftverfejlesztés egy olyan széles körben elterjedt technológia, mely elősegíti az összetett szoftverrendszerek tervezését, készítését, továbbfejlesztését és karbantartását.
A technológia domének modellezésére és a modellek automatizált transzformációjára épít.

A szakdolgozat elkészítése során megismerkedtem a modellezés és modell-vezérelt szoftverfejlesztés technikáival, és az ezt támogató Eclipse integrált fejlesztői környezettel, és Eclipse Modeling Framework technológiával.

Megismertem a statikus analízis módszereit és lehetőségeit a szoftverfejlesztés hatékonyságának javítására.
  lekérdezések szöveges leírására szolgáló nyelvével és annak használatával, és az Xtext validátorok írásával.
Elemeztem a nyelvben statikus analízis segítségével megoldható problémák témakörét, és kiválasztottam ezek közül egy gyakran előforduló és kellemetlen következményekkel járó problémát, melynek kezelése mégis egyszerű, \sout{a féléves munka során megvalósítható}.
A probléma okának és körülményeinek mélyebb megismerése után megterveztem és megvalósítottam a megoldáshoz szükséges ellenőrzést.
Az elkészült alkotást működés közben a ~\ref{fig:unusedLive}. ábra mutatja.
\sout{Ugyan éles bevetésére még nem került sor}, az EMF-IncQuery fejlesztői és felhasználói már most élénk érdeklődést mutatnak iránta.

Az alábbi célokat értem el: ...

% * Eclipse contrib
% * GitHub források

\section{Kitekintés}

% * több törzs
% * psystem
% * többi jelleg (kind)
% * sok egyéb statikus analízis