%----------------------------------------------------------------------------
% Abstract in hungarian
%----------------------------------------------------------------------------
\chapter*{Kivonat}\addcontentsline{toc}{chapter}{Kivonat}

A dolgozat a modell-vezérelt szoftverfejlesztés során, modell-lekérdezések definiálásakor előforduló hibalehetőségek eseteit vizsgálja, konkrét környezetként az Eclipse Modeling Framework (EMF) platformjára épülő, a BME Méréstechnika és Információs Rendszerek Tanszékének Hibatűrő Rendszerek Kutatócsoportja által kidolgozott EMF-IncQuery lekérdező rendszer esetére.
Ez a rendszer EMF modellek felett teszi lehetővé gráfminta alapú deklaratív lekérdezések elvégzését.

A dolgozat lekérdezéseknek az EMF-IncQuery tárgynyelvén történő megfogalmazása során elkövethető emberi hibák elkerülése céljából a lekérdezőnyelv statikus analízisével foglalkozik.
A statikus analízis segítségével, fejlesztési időben rendelkezésre álló adatok alapján elvégezhető vizsgálatokkal hívja fel a fejlesztők figyelmét a lekérdezésekben található hibákra, hibalehetőségekre.

A dolgozat a tématerületet ismertető első fejezet után a második fejezetben beágyazza az EMF-IncQuery keretrendszert az EMF kontextusába, majd példákon keresztül ismerteti annak használatát és bemutatja lekérdezőnyelvét.

A harmadik fejezet olyan ellenőrzések megtervezését és a kivitelező komponens elkészítését mutatja be, amelyek az EMF-IncQuery lekérdezőnyelvében található minták változóira történő hivatkozások minőségi és mennyiségi elemzése alapján tárják fel a fejlesztői hibalehetőségeket.

A negyedik fejezet a lekérdezések és a kapcsolódó domén-modell közötti lehetséges ellentmondásokat a mintákban található változók között fennálló funkcionális függőségeken keresztül ellenőrző statikus vizsgálatokat mutatja be.
Végül egy esettanulmányon keresztül szemlélteti a fejezetben ismertetett elemzések elvégzését segítő funkcionális függőség elemző kód felhasználását a lekérdezések végrehajtásának hatékonyabbá tételére.

A dolgozatot az elkészült komponensek továbbfejlesztési lehetőségeinek áttekintése zárja.

\vfill

%----------------------------------------------------------------------------
% Abstract in english
%----------------------------------------------------------------------------
\chapter*{Abstract}\addcontentsline{toc}{chapter}{Abstract}

The thesis analyzes the possible mistakes made in the definitions of model queries in context of model-driven software development.
As a concrete environment the EMF-IncQuery system has been selected that was developed by the Fault Tolerant Systems Research Group of the Department of Measurement and Information Systems at Budapest University of Technology and Economics.
This system enables the performance of graph pattern based queries on EMF models.

The thesis deals with static analysis of the query language of EMF-IncQuery to avoid human mistakes that can be made while composing queries using the language.
With the help of static analysis, it draws developers' attention to errors and possible mistakes occurring in queries with inspections based on data available at development time.

The thesis, after introducing the topic in the first chapter, in the second chapter embeds the EMF-IncQuery framework into the context of EMF then demonstrates its usage through examples and presents its query language.

The third chapter introduces the design and implementation of such an inspection component that reveals developers' mistakes based on the qualitative and quantitative analysis of references to patterns' variables.

The fourth chapter introduces static analyses that check for possible inconsistencies between the query and the associated domain-model through the functional dependencies existing among the variables of the patterns.
Finally demonstrates, through a case study, the use of the functional dependency checking code that supported the performance of analyses presented in the chapter for improving the efficiency of query execution. 

The thesis closes with the review of possible further development of the realized components.

\vfill
